\section{Introduction} \label{sec:intro}

In this assignment we used the OpenVAS vulnerability scanning tool to gather information about and to assess the security of a system (remote computer) used for the identification and correction of security flaws. Vulnerability Assessment is a technique used to evaluate resources and assets present in a company. This type of audit is based on the identification of open ports, available services and from this the detection of possible failures present in the target systems. The purpose is to know what vulnerabilities exist in a company's systems and thereby develop an appropriate action plan.
In this way, a security assessment can be carried out on the systems of an organisation in order to increase security in them. In addition to this activity, there are other concerns, so it is necessary to complement it with security solutions, such as those against malicious code, firewalls, intrusion detection tools and good security policies contribute to the protection of asset and the combination of different approaches to security. 
First of all we had to get familiar with how OpenVas works (interface as well) and try to understand the framework and it's different options and parameters. Then it comes the analysis part, because is really important to understand the different vulnerabilities that a system can has as well as find a way to solve them making the system more secure.  
\newline

